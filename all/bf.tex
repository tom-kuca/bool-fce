\input ../macros.tex

% Sazba obsahu
\newwrite\toc
\immediate\openout\toc=bf.toc
\def\writetoc#1#2{\write\toc{\string\tocentry{#1}{#2}{\the\count0}}}
\def\tocout{
\immediate\closeout\toc
\immediate\openin\toc=bf.toc
\ifeof\toc
\else
\input bf.toc
\fi
\immediate\closein\toc
}

% Ujisti se, ze na strance zbyva dost mista
\def\prechapter#1#2{%
\checkroom{30pt}
\writetoc{#1}{#2}
}
\def\chapterend{\bigbreak\bigbreak}

{\nopagenumbers
\vglue 1cm
\centerline{\Large }
\bigskip
\bigskip
\centerline{\big Boolovsk� funkce a jejich aplikace}
\bigskip
\bigskip
\centerline{\large dle p�edn�ky doc. RNDr. Ond�ej �epka, Ph.D.}
\centerline{\large ITI 2010}
\vskip 0.5cm
\eject

\bigskip
\eject
}

{\nopagenumbers

Z�pisky z p�edm�tu Boolovsk� funkce a jejich aplikace z roku 2010, semin�� vede
doc. RNDr. Ond�ej �epek, Ph.D. Sepsali Tom� Ku�a a Jind�ich Vodr�ka. Z�pisky
nejsou aktualizov�ny, mohou obsahovat chyby. Z�skat zdrojov� \TeX k�dy m��ete
v git repository { \tt https://github.com/tom-kuca/bool-fce}. Tamt� lze
hl�sit chyby, p��padn� p�isp�t.

\vfill

\eject

}



\input body.tex

\vfill\eject

~~~

\vfill\eject

\def\writetoc#1#2{}
\prednaska{O}{Obsah}{}

\def\tocentry#1#2#3{\line{\hbox to 1.5em{\hfil #1.} #2\dotfill #3}\vskip 2pt}
\tocout
\tocentry{O}{Obsah}{\the\count0}

\vfill\eject
\nopagenumbers


\eject\end
