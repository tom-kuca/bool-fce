\input ucwmac.tex
\input ucw-ofs.tex
\input ucw-verb.tex
\input fonts12.tex
\input epsf.tex

\def\em#1{{\it #1\/}}
\def\df#1{{\it #1\/}}	% when we define something
\def\O{{\cal O}}
\def\F{{\cal F}}
\let\>=\noindent
\def\TODO#1{\>{\bf TODO:} #1}
\def\FIXME#1{\>{\bf TODO:} #1}
\def\qed{{\parfillskip=0pt\allowbreak\hfill\nobreak $\spadesuit$\par}}
\def\qeditem{{\parfillskip=0pt\hfill\rlap{\hskip\rightskip\llap{$\spadesuit$}}\par}}
\def\qedmath{\eqno{\spadesuit}}
\def\symdiff{\mathbin{\Delta}}
\def\rack#1#2{\setbox0=\hbox{#1}\hbox to \wd0{#2}}
\def\o#1{\accent23 #1}
\def\mst{\mathop{\rm mst}}
\def\msf{\mathop{\rm msf}}
\def\deg{\mathop{\rm deg}}
\def\timesalpha{\mskip2mu\alpha}
\def\timesbeta{\mskip2mu\beta}
\def\tower{\mathbin\uparrow}
\def\Forb{{\rm Forb}}
\def\minorof{\preccurlyeq}
\def\per{\mathop{\rm per}}
\def\poly{\mathop{\rm poly}}
\def\E{{\bb E}}
\def\N{{\bb N}}
\def\Z{{\bb Z}}
\def\crpt{\mathbin{\Uparrow}}
\def\C{{\cal C}}
\def\sgc{\mathbin{.}}
\def\veclength#1{\vert \vec{#1} \vert}
\def\distance#1#2{\veclength{#1 #2}}
\def\O{{\cal O}}                % Asymptoticke O-cko
\def\assign{\leftarrow} 

\def\implication{\Rightarrow}

\def\invimplication{\Leftarrow}

% Zacatek prednasky {cislo prednasky}{jmeno prednasky}{jmeno zapisovatele}
\def\prednaska#1#2#3{%
\prechapter{#1}{#2}
\thmcount=0
\footcnt=0
\edef\currentchap{#1}
\edef\currentid{\currentchap.\the\thmcount}
\vbox{%
\line{{\Large\bf #1. #2} \hfil {\it #3}}
\vskip 4pt
\hrule}
\medskip
}
\def\prechapter#1#2{}

% Nadpis {text}
\def\h#1{\medbreak\leftline{\bf #1}\nobreak\smallskip\nobreak}

% Zvyrazneny zacatek odstavce coby podnadpis (napr. vety apod.)
\def\s#1{\noindent {\bo #1}}

% A kdyz stoji samostatne (aby se naodlamoval)
\def\ss#1{\noindent {\bo #1}\par\nobreak}

% Poznamky pod carou
\newcount\footcnt
\footcnt=0
\def\foot#1{\global\advance\footcnt by 1{\parindent=0.25in\parskip=0pt\footnote{$^{\left<\the\footcnt\right>}$}{#1}}}

% \noindent se casto hodi, tak na nej mame zkratku
\let\>=\noindent

% Vlozeni obrazku {obrazek}{popisek}{sirka}
\def\figure#1#2#3{\bigskip\vbox{\centerline{\epsfxsize=#3\epsfbox{#1}}\smallskip\centerline{#2}}\bigskip}

% Varianta bez popisku
\def\fig#1#2{\medskip\centerline{\epsfxsize=#2\epsfbox{#1}}\medskip}

% Dva obrazky vedle sebe s popiskami
\def\twofigures#1#2#3#4#5#6{\bigskip\centerline{\vbox{\halign{\hfil##\hfil\hskip 4em&\hfil##\hfil\cr
\epsfxsize=#3\epsfbox{#1}&\epsfxsize=#6\epsfbox{#4}\cr
\noalign{\smallskip}
#2&#5\cr}}}\bigskip}

% Obrazek vlozeny do praveho okraje odstavce {obrazek}{sirka}
% Pouzit na zacatku odstavce a nejlepe celou konstrukci zavrit do vboxu, aby se nerozlomila
\def\inlinefig#1#2{
\setbox0=\hbox{\epsfxsize=#2\epsfbox{#1}}
\hangindent=-\wd0
\advance\hangindent by -3em
\dimen0=\ht0
\advance\dimen0 by 8ex
\advance\dimen0 by \normalbaselineskip
\count0=\dimen0
\divide\count0 by \normalbaselineskip
\hangafter=-\count0
\dimen0=\normalbaselineskip
\multiply\dimen0 by \count0
\vbox to 0pt{}
\nointerlineskip
\vbox to 0pt{\vbox to \dimen0{\vss\rightline{\box0\hskip 1em}\vss}\vss}
\nointerlineskip
}

% Todo
\def\todo#1{{\bf TODO: \it #1}}


% Moving the tilda to the correct height for a url
\def\urltilda{\kern -.15em\lower .7ex\hbox{\~{}}\kern .04em}
\def\urldot{\kern -.10em.\kern -.10em}
\def\urlhttp{http\kern -.10em\lower -.1ex\hbox{:}\kern -.12em\lower 0ex\hbox{/}\kern -.18em\lower 0ex\hbox{/}}

\def\brk{\hfil\break}




% Bit strings
\def\0{{\bf 0}}
\def\1{{\bf 1}}
\def\(#1){\mathord{\left<#1\right>}}

% Bitwise operations
\def\shl{\mathbin{<\!<}}
\def\shr{\mathbin{>\!>}}
\def\bop#1{\mathbin{\hbox{\sc #1}}}
\def\band{\bop{and}}
\def\bor{\bop{or}}
\def\bxor{\bop{xor}}
\def\bnot{\mathop{\hbox{\sc not}}}

% Footnotes
\newcount\footcnt
\footcnt=0
\def\foot#1{\global\advance\footcnt by 1{\parindent=0.25in\parskip=0pt\footnote{$^{\bf\the\footcnt}$}{#1}}}

%%% Chapters, sections and proclamations %%%

\newcount\chapcount
\newcount\thmcount
\newcount\tmpcount
\newcount\figcount
\chapcount=0
\thmcount=0
\figcount=0
\def\currentid{??}
\def\currentchap{??}

\def\appendices{\chapcount=99}

\def\link#1{\mdwsmall\tt #1}

\def\para{\smallskip\noindent}

\def\fignext{\advance\figcount by 1
\edef\currentid{\currentchap.\the\figcount}
}

\def\thmnext{\advance\thmcount by 1
\edef\currentid{\currentchap.\the\thmcount}
}


\def\proclaim#1{\para {\bf #1 \enspace}}
\def\fignum#1{\fignext\noindent {Obr\'{a}zek \currentid.\enspace} {#1.\enspace}}

\def\thm{\thmnext\proclaim{V\v{e}ta \currentid}}
\def\lemma{\proclaim{Lemma}}
\def\claim{\proclaim{Tvrzen\'\i{}}}
\def\defn{\proclaim{Definice}}
\def\problem{\proclaim{Probl\'em}}
\def\obs{\proclaim{Pozorov\'an\'\i{}}}
\def\rem{\proclaim{Pozn\'amka}}
\def\alg{\proclaim{Algoritmus}}
\def\impl{\proclaim{Implementace}}
\def\cor{\proclaim{D\o{u}sledek}}
\def\nota{\proclaim{Zna\v{c}en\'\i{}}}
\def\example{\proclaim{P\v{r}\'\i{}klad}}

\def\paran#1{\para {\sl #1.\/}\enspace\eatspaces}

\def\proof{\noindent {\sl D\o{u}kaz.}\enspace}
\def\proofsketch{\noindent {\sl Idea d\o{u}kazu.}\enspace}


\def\label#1{{\sl (#1)\/}\enspace}
\def\labelx#1{\label{#1}\hfil\break\eatspaces}
\def\eatspaces{\kern0pt}

\def\thmn{\thm\labelx}
\def\lemman{\lemma\labelx}
\def\defnn{\defn\labelx}
\def\corn{\cor\labelx}
\def\algn{\alg\label}
\def\notan{\nota\labelx}
\def\examplen{\example\labelx}
\def\problemn{\problem\labelx}
\def\remn{\rem\labelx}

%%% Algorithms %%%

%	\algo{jmeno algoritmu}
%	\algin popis vstupu
%	\:krok
%	\:krok
%	\::vnoreny krok
%	\algout popis vystupu
%	\endalgo
%
\def\interlistskip{\vskip 3pt plus 2pt minus 1pt}


% Nadpis {text}
\def\sec#1{\bigskip\leftline{\large\bf #1. }\nobreak\bigskip\nobreak}

\def\paran#1{\para {\sl #1.\/}\enspace\eatspaces}

\def\algo{
\interlistskip
\begingroup
\let\:=\algoitem
\parskip=1pt plus 1pt minus 0.3pt
\rightskip=2em
\itemcount=0
\smallskip
\algn
}
\def\endalgo{\interlistskip\endgroup}
\def\algopar{\par
\parindent=2em
\hangindent=4em
\hangafter=1
\leavevmode
}
\def\algoitem{
\algopar\advance\itemcount by 1
\hbox to 2em{\hss \the\itemcount. }%
\futurelet\next\algoitemh}
\def\algoitemh{\ifx\next:\let\next=\algohang\else\let\next=\relax\fi\next}
\def\algohang:{\advance\hangindent by 2em \hskip 2em\futurelet\next\algoitemh}
\def\algin{\par{\sl Vstup:\/} }
\def\algout{\par{\sl V\'ystup:\/} }

%%% Constructs used in algorithms %%%

\def\={\leftarrow}
\def\cmt#1{~~{\sl (#1)}}

%%% Enumerated lists %%%

\def\eqalign#1{\null\,\vcenter{\openup\jot
  \ialign{\strut\hfil$\displaystyle{##}$&$\displaystyle{{}##}$\hfil
      \crcr#1\crcr}}\,}

%%% References %%%

\newwrite\ids
\def\writeid#1#2{\immediate\write\ids{\string\def\expandafter\string\csname id#1\endcsname{#2}}}

\immediate\openin\ids=\jobname.ids
\ifeof\ids
\else
\input \jobname.ids
\fi
\immediate\closein\ids
\immediate\openout\ids=\jobname.ids

\def\ref#1{\expandafter\ifx\csname id#1\endcsname\relax
{\bf ??}%
\immediate\write16{*** Warning: Reference id#1 undefined ***}%
\else
\csname id#1\endcsname
\fi
}

\def\id#1{\writeid{#1}{\currentid}}

\def\strutA#1#2{\vrule height#1 depth#2 width0pt}

% Ujisti se, ze na strance je dostatek mista, pripadne zacne novou stranku
\def\checkroom#1{\vskip 0pt plus #1\goodbreak\vskip 0pt plus -#1}
